\chapter{Introduction}

\paragraph{}eSim is a CAD tool that helps electronic system designers to design, test and analyse
their circuits. But the important feature of this tool is that it is open source and hence
the user can modify the source as per his/her need. The software provides a generic,
modular and extensible platform for experiment with electronic circuits. This software
runs on all Ubuntu Linux distributions and some flavours of Windows. It uses Python,
KiCad and Ngspice.

\paragraph{}

The objective behind the development of eSim is to provide an open source EDA
solution for electronics and electrical engineers. The software should be capable of
performing schematic creation, PCB design and circuit simulation (analog, digital and
mixed signal). It should provide facilities to create new models and components. The
architecture of eSim has been designed by keeping these objectives in mind \cite{esimusermanual} .

\subsection{Workflow of eSim}
\paragraph{}

This section describes the workflow of eSim for circuit simulation. The flow would be slightly different for PCB designing. For a generic treatment on the workflow, please refer to the user manual .

\paragraph{}

The first step in circuit simulation is to draw a schematic diagram of the circuit. In eSim it is drawn using Eeschema. The next step is to obtain a netlist file. The schematic drawing tool provides this file, which describes the electrical connections between components.
The netlist generated by Schematic Editor cannot be directly used for simulation due to compatibility issues. Netlist Converter converts it into Ngspice compatible format.The type of simulation to be performed and the corresponding options are provided through a graphical user interface (GUI). This is called KiCad to Ngspice Converter in eSim. 
eSim uses Ngspice for analog, digital, mixed-level/mixed-signal circuit simulation.


\paragraph{}To summarise the simulation workflow:

\begin{enumerate}
\item
Strat a new project in the eSim GUI.
\item
Draw the schemtic diagram of the circuit in Eeschema.
\item
Convert it to netlist for simulation using the schematic drawing tool.
\item
Come back to the eSim GUI and convert the netlist for use by Ngspice using KiCad to Ngspice Converter. Give details of the type of simulations during this step.

\item
Simulate the circuit and plot the required variables.

\end{enumerate}


\paragraph{}

The minutest details on every step given above is avilable in the esim usermanual\cite{esimusermanual}. Feel free to go through it.



